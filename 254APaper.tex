\documentclass{article}

\usepackage[english]{babel}
\usepackage{amsmath}
\usepackage{amssymb}
\usepackage{amsthm}

\usepackage[letterpaper,top=2cm,bottom=2cm,left=3cm,right=3cm,marginparwidth=1.75cm]{geometry}
\usepackage{graphicx}
\usepackage[colorlinks=true, allcolors=blue]{hyperref}
\usepackage{fancyhdr}
\usepackage{tikz}
\usetikzlibrary{decorations.markings,calc}
\usepackage{tikz-cd}
\usepackage{quiver}
\usetikzlibrary{matrix}
\usepackage[most]{tcolorbox}
\usepackage{hyperref}
\usepackage{array}
\usepackage{colonequals}
\usepackage{todonotes}
\usepackage{cleveref}

\font\maljapanese=dmjhira at 2.5ex
\newcommand{\yo}{\textrm{\!\maljapanese\char"48}}

\newtheorem{theorem}{Theorem}[section]

\theoremstyle{definition}
\newtheorem{definition2}[theorem]{Definition}


\newtheorem{definition}[theorem]{Definition}
\newtheorem{example}[theorem]{Example}
\newtheorem{examples}[theorem]{Examples}


\theoremstyle{remark}
\newtheorem*{remark}{Remark}
\newtheorem*{notation}{Notation}

\theoremstyle{plain}
\newtheorem{proposition}[theorem]{Proposition}
\newtheorem{conjecture}[theorem]{Conjecture}
\newtheorem{corollary}[theorem]{Corollary}
\newtheorem{lemma}[theorem]{Lemma}

\newcommand{\R}{\mathbb{R}}
\newcommand{\C}{\mathbb{C}}
\newcommand{\Z}{\mathbb{Z}}
\newcommand{\N}{\mathbb{N}}
\newcommand{\Q}{\mathbb{Q}}
\newcommand{\mb}[1]{\mathbb{#1}}
\newcommand{\mc}[1]{\mathcal{#1}}
\newcommand{\mk}[1]{\mathfrak{#1}}
\newcommand{\m}{\mathfrak{m}}
\newcommand{\un}{\cup}
\newcommand{\ic}{\cap}
\newcommand{\qa}[2]{(\frac{#1, #2}{K})}
\pagestyle{fancy}
\newcommand\size{1}% distance of nodes from center

\usepackage{microtype}

\usepackage{caption}
\captionsetup[figure]{labelformat=empty}%

\begin{document}

\title{The Hasse-Minkowski Theorem and Local-Global Principles}

\maketitle

\tableofcontents

\section{Introduction}
\todo{add brief introduction to local/global principles}
\todo{cite theorems after I say what they do}



\section{Quadratic Spaces}

We will work over an arbitrary field $K$ in this section, assumed to be of characteristic different from 2. \newline

\indent	A \textit{quadratic form} over a field $K$ is a homogeneous polynomial in $n$ variables over $K$, which is homogeneous of degree 2.
A quadratic form $f$ can be expressed as $f(X_1, \dots, X_n) = \sum_{i,j = 1}^n a_{ij} X_i X_j$
where we may take the averages of $a_{ij}$ and $a_{ji}$ (as char$(K)$ is not 2) and assume they are equal, so that $f$ determines a unique symmetric matrix $M_f = (a_{ij})$.
In this way, letting $X = (X_1, \dots, X_n)$ we have, in matrix notation that $f(X) = X^T  M_f  X$.\newline

\indent Two quadratic forms $f,g$ are said to be \textit{equivalent} (which we denote as $f \sim g$) if there exists an invertiible matrix $C \in \text{GL}_n(K)$ such that $f(X) = g(C  X)$.
Now since 
\[g(C  X) = (C X)^T M_f (C  X) = X^T  (C^T  M_g  C)  X\]
equivalence of quadratic forms amounts to congruence of their associated matrices $M_f$ up to transformations of the form $C^T  M_g \ C, C \in \text{GL}_n(K)$.

\begin{example}\label{ex:hyperbolic-plane}
	Consider the quadratic form $f = X_1X_2$ in two variables. 
	Under the linear change of coordinates $X_1 =  Y_1  + Y_2, X_2 = Y_1 - Y_2$ we see our new equivalent form expressed in terms of $Y_1, Y_2$ is $(Y_1 + Y_2)(Y_1 - Y_2)  = Y_1^2 - Y_2^2$.
\end{example}

We now give a more geometric way of interpreting quadratic forms by defining quadratic spaces. 

\begin{definition}[Quadratic Space]
	A \textit{quadratic space} is a pair $(V,B)$ where $V$ is a finite-dimensional vector space over $K$, and $B: V \times V \to K$ a symmetric bilinear form.
\end{definition}

The bilinear pairing $B$ defines a map $Q: V \to K$ by sending $v \in V$ to $B(v,v)$.
It is in fact a \textit{quadratic map} as it satisfies 
\[Q(ax) = B(av, av) = a^2 B(v,v) = a^2 Q(x)\]
and 
\[Q(x+y) - Q(x) - Q(y) = B(x+y,x+y) - B(x,x) - B(y,y) = 2 B(x,y)\]
is a bilinear form.
Then it is clear that $Q$ also determines $B$ via $B = \frac{1}{2}(Q(x+y) - Q(x) - Q(y))$, so we may also equivalently write $(V,Q)$ for the corresponding quadratic space. \newline

\indent Choosing a basis $e_1, \dots, e_n$ for $V$ defines a quadratic form $f = \sum_{i,j = 1}^n B(e_i, e_j) X_i X_j$ for which evaluation at $x \in K^n$ is exactly our quadratic map $Q$.
Any different choice of basis $e'_1, \dots, e'_n$ is related by a change of coordinates matrix $C$, so its corresponding quadratic form $f'$ is equivalent to $f$. 
Hence a quadratic space uniquely determines an equivalence class of quadratic forms which we denote as $(f_B)$. Conversely, any $n$-ary quadratic form determines a quadratic space over $K^n$ via its natural evaluation map. 
 

\begin{definition}[Metric Morphism]
	A map $\phi: (V,B) \to (V', B')$ of quadratic spaces is called a \textit{metric morphism} if $\forall x,y \in V$ we have $B'(\phi(x), \phi(y)) = B(x,y)$.
\end{definition}

\todo{Remark on name?}
This naturally defines a category whose objects are quadratic spaces over $K$, and morphisms being metric morphisms. If $(V,B)$ and $(V',B')$ are isomorphic in this category (which we denote as $(V,B) \cong (V',B'))$, it is clear from our above correspondence that $(f_B) \cong (f_{B'})$.
From now on in this section using this equivalence, we will tend to think more coordinate-freely through quadratic spaces, whose results then can be translated as above to those of quadratic forms.

\begin{definition}[Nondegeneracy]
	For $(V,B)$ a quadratic space, and $M$ the corresponding symmetric matrix to some form in the eqivalence class $(f_B)$, then $(V,B)$ is called \textit{nondegenerate} if the following equivalent conditions hold
	\begin{enumerate}
		\item $M$ is an invertible matrix.
		\item $B$ defines a perfect pairing, i.e. the map $x \to B(-,x)$ defines an isomorphism between $V$ and its dual space $V^*$
	\end{enumerate}	
\end{definition}

Now note that any subspace $S$ of $V$ has, by restriction of $B$ (equivalently $Q$), the structure of a quadratic space.
The \textit{orthogonal complement} of $S$ is defined as usual
\[S^{\perp} = \{v \in V \,|\, B(v,S) = 0\}\]
The orthogonal complement of $V$ itself, $V^{\perp} = \text{rad } V$ is called the \textit{radical} of $V$.

\begin{proposition}\label{prop:nondeg-ss-dim}
	Let $(V,B)$ be a nondegenerate quadratic space and $S$ a subspace of $V$.
	Then
	\begin{enumerate}
		\item $\text{dim}(S) + \text{dim}(S^{\perp}) = \text{dim}(V)$
		\item $(S^{\perp})^{\perp} = S$
	\end{enumerate}
	
\end{proposition}
\begin{proof}
	From the isomorphism $\phi: V \to V^*$ defined by the pairing $B$, note that $S^{\perp}$ is precisely the subspace of functionals which annihilate $S$.
	By duality, we have that 
	\[\text{dim}(S^{\perp}) = \text{dim}(V^*) - \text{dim}(\phi(S)) = \text{dim}(V) - \text{dim}(S)\]
	since $\phi$ is an isomorphism, establishing the first result. \newline

	\indent For the second result, it suffices to note that $(S^{\perp}))^{\perp} \subset S$ (under the usual canonical isomorphism $V \cong (V^*)^*$ and each $s \in S$ is 0 under those functionals in $S^{\perp}$).
	However, from the first results we have 
	\[\text{dim}((S^{\perp}))^{\perp}) = \text{dim}(V) - \text{dim}(S^{\perp}) = \text{dim}(V) - (\text{dim(V)} - \text{dim}(S)) = \text{dim}(S)\]
	so they must be equal as their dimensions are equal and one is contained in the other.
\end{proof}

\subsection{Diagonalization of Quadratic Forms}

In this section, we will be giving an useful representative of each equivalence class of quadratic forms, established via an "orthogonal basis" with respect to a bilinear form.

\begin{definition}[Representability]
	Let $f$ be a quadratic form over $K$, and $d \in K^{\times}$.
Then we say $f$ \textit{represents} $d$ if there exists some (necessarily nonzero) $X \in K^n$ such that $f(X) = d$.
\end{definition}

Note that the set of representable values of a quadratic form only depends on its equivalence class, so we can further say a quadratic space $(V,B)$ represents $d$ if any of its associated quadratic forms represents $d$.

\begin{definition}[Orthogonal Sums]
	If $(V_1, B_1)$ and $(V_2, B_2)$ are quadratic spaces, then we may form their \textit{orthogonal sum}, a quadratic space $V_1 \perp V_2 = (V_1 \oplus V_2, B)$ where $B$ is the pairing given by $B((x_1, x_2), (y_1, y_2)) = B_1(x_1, y_1) + B_2(x_2, y_2)$.
\end{definition}

\begin{remark}
	The corresponding quadratic map of $V_1 \perp V_2$ satisfies $Q(x_1, x_2) = B((x_1, x_2), (x_1, x_2)) = B_1(x_1, x_2) + B_2(x_1, x_2) = Q_{B_1}(x_1) + Q({B_2}(x_2)$.
Choosing a basis for $V$ (so defining a quadratic form), we see that the notion of orthogonal sum for quadratic forms $f$ and $g$ corresponds to adjoining them in separate variables.
We will refer to this operation as $f + g$.
\end{remark}

\begin{notation}
	We shall denote the $\langle d \rangle$ for the isometry class of 1-dimensional quadratic spaces corresponding to the quadratic form $d X^2$.
\end{notation}

\begin{proposition}\label{prop:rep-criterion1}
	Let $(V,B)$ be a quadratic space, and $d \in K^{\times}$. Then $(V,B)$ represents $d$ if and only if there exists another space $(V',B')$ such that $V \cong V' \perp \langle d \rangle$. 
\end{proposition}

\begin{proof}
	If $V \cong V' \perp \langle d \rangle$ then $V$ represents $d$ as $\langle d \rangle$ clearly does. \newline
	\indent Conversely, assume that $V$ respresents $d$.
	We first show that we may assume that $V$ is non-degenerate.
	Indeed, consider any subspace $W \subset V$ such that $W \oplus \text{rad} V = V$.
	Since $\text{rad}(V)$ is orthogonal to all of $V$ we have furthermore that $V = W \perp \text{rad}(V)$.
	As noted, the quadratic map associated to $V$ are then defined as the composites of those associated to $W$ and $\text{rad}(V)$ (the latter identically zero), so any quadratic form associated to $V$ takes nonzero values only on $W$.
	Hence we may restrict ourself to $W$ which is non-degenerate. \newline
	\indent Now since $W$ represents $d$, there exists $v \in W$ such that $Q(v) = d$.
	Then the $K \cdot v \cong \langle d \rangle$ ($K \cdot v$ denotes the one-dimensional subspace spanned by $v$).
	As $d \neq 0$, no nonzero element in $K \cdot v$ is orthogonal to another, hence $(K \cdot v) \ic (K \cdot v)^{\perp} = 0$.
	But by \Cref{prop:nondeg-ss-dim} we have $\text{dim}(K \cdot v) + \text{dim}((K \cdot v)^{\perp}) = \text{dim}(W)$ so therefore 
	\[W = (K \cdot v)^{\perp} \perp (K \cdot v) \cong (K \cdot v)^{\perp} \perp \langle d \rangle\]
	as wanted.
\end{proof}

Translating this into the language of quadratic forms we obtain the following corollary.

\begin{corollary}\label{coroll:rep-criterion1}
	If a quadratic form $f$ represents $d$, then there exists a quadratic form $f'$ such that $f \sim f' + dZ^2$
\end{corollary}

\begin{corollary}\label{coroll:orthogonal-basis}
	If $(V,B)$ is any quadratic space, then there exists $d_1, \dots, d_n \in K$ such that $V \cong \langle d_1 \rangle \perp \dots \perp \langle d_n \rangle$. 
\end{corollary}
 \begin{proof}
	We induct on the dimension of $V$.
	It is trivial when $V$ is zero-dimensional.
	Otherwise, suppose $\text{dim}(V) = n > 0$.
	If $V$ doesn't represent any non-zero values it is the orthogonal sum of $\langle 0 \rangle$'s.
	Otherwise, it represents some $d \in K^{\times}$ and so by \Cref{prop:rep-criterion1} we have $V \cong \langle d \rangle \perp V'$, for $V'$ of smaller dimension.
	Thus if $V' = \langle d_1 \rangle \perp  \dots\perp \langle d_{n-1} \rangle$ from our induction hypothesis we have that $V \cong \langle d \rangle \dots \langle d_1\rangle \perp \dots \perp \langle d_{n-1} \rangle$ as desired.
 \end{proof}

From the equivalence between isomorphism classes of quadratic space and equivalence classes of quadratic forms, the above theorem gives us the folllowing result for quadratic forms. 

 \begin{corollary}
 	Any quadratic form $f$ is equivalent to one of the form $d_1 X_1^2 + \dots + d_n X_n^2$
 \end{corollary}
 
 \begin{remark}
 	The $d_1, \dots, d_n$ are not unique, even up to reordering. However, one can show they are unique up to something called chain equivalence, see \cite{ATOQF}, I.5 for details.
 \end{remark}
 
 \begin{notation}
 	We shall denote $\langle d_1 \rangle \perp \dots \perp \langle d_n \rangle$ as $\langle d_1, \dots, d_n \rangle$.
	If all the $d_i$ are nonzero, we shall say the corresponding quadratic form $f = d_1 X_1^2 + \dots + f_n X_n^2$ is of \textit{rank} $n$.
\end{notation}

 \begin{proposition}\label{prop:nondegen-decomp}
	If $(V,B)$ is a quadratic space and $S$ a nondegenerate subspace, then $V = S \perp S^{\perp}$	
\end{proposition}
 
\begin{proof}
	Note that $S \ic S^{\perp}$ is from definitions $\text{rad }(S)$, but as $S$ is nondegenerate we have that $S \ic S^{\perp} = 0$.
	So it suffices to show that $S \oplus S^{\perp} = V$.
	From \Cref{coroll:orthogonal-basis} we can choose an orthogonal basis $s_1, \dots, s_k$ for $S$ 
	For any $v \in V$ we explicitly construct the "orthogonal projection" of $v$ onto $S$, which we define as 
	\[y = v - \sum_{i = 1}^{k} \frac{B(v, s_i)}{B(s_i, s_i)} s_i\]
Then $y$ is orthogonal to $s$, as for any $s_j$ we explicitly verify that 
\begin{align*}
	B(y, s_j) = B(v, s_j) - \sum_{i = 1}^{k} \frac{B(v, s_j)}{B(s_i, s_i)} B(s_i, s_j) = B(v, s_i) -  \frac{B(s_j, s_j)}{B(s_j, s_j)}B(v,s_i) = 0 
\end{align*}
as the $s_i$ are all orthogonal to each other. 
Thus $y \in S^{\perp}$ and $v = y + \sum_{i = 1}^{k} \frac{B(v, s_i)}{B(s_i, s_i)} s_i \in S \oplus S^{\perp}$ shows they span $V$ as we want.
\end{proof}

We are now in good shape to define a major invariant associated to a quadratic form.

\begin{definition}[Discriminant]
	For a quadratic form $f$, let $M_f$ be its correpsonding symmetric matrix. 
	Then the \textit{discriminant} $d(f)$ is defined as 
	\[d(f) = \text{det}(M_f) \in K^{\times}/ (K^{\times})^2\]
\end{definition}

We note that the discriminant is indeed well-defined in $K^{\times}/ (K^{\times})^2$ (The \textit{group of square classes}) as for any $M_g$ for an equivalent quadratic form $g$ we have 
\[\text{det}(M_g) = \text{det}(C^T M_f C) = \text{det}(C)^2 d(f)\]
for some invertible matrix $C$, hence it is a well-defined element in the group of square classes.
We may then also talk about the discriminant of a quadratic space $d(V)$ from its corresponding equivalence class of quadratic forms.\newline

\indent For a quadratic form in a diagonal form $f =  d_1 X_1^2 + \dots + d_n X_n^2$ the matrix $M_f$ is diagonal with entries $d_i$, so its discriminant is easily calculated as $d_1 \dots d_n \in K^{\times}/ (K^{\times})^2$.

 \subsection{The Hyperbolic Plane}


 \begin{definition}
 	We call a nonzero vector $v$ in a quadratic space $(V,B)$ \textit{isotropic} if $Q(v) = B(v,v) = 0$, otherwise $v$ is \textit{anisotropic}.
	The quadratic space $(V,B)$ is \textit{isotropic} if it contains an isotropic vector, otherwise it is \textit{anisotropic}.
 \end{definition}

 \begin{theorem}[Hyperbolic Plane]\label{theo:hyperbolic-plane}
 	Let $(V,B)$ be a 2-dimensional quadratic space, then the following are equivalent
	\begin{enumerate}
		\item $V$ is nondegenerate and isotropic
		\item $V$ is regular, with $d(V) = -1$
		\item $V \cong \langle 1, -1 \rangle$ 
	\end{enumerate}
	The isometry class of such spaces is called the \textit{hyperbolic plane}, which we denote as $\mb{H}$.
 \end{theorem}

 \begin{proof}
	We first show that $1. \to 2.$. Choose $e_1, e_2$ an orthogonal basis for $V$. 
	As $V$ is nondegenerate we have that both $Q(e_1) = d_1$ and $Q(e_) = d_2$ are both nonzero.
	As $V$ is isotropic, there exists an isotropic vector $ae_1 + be_2$.
	Then 
	\begin{align*}
		0 = Q(ae_1 + be_2) = ad_1^2 + bd_2^2 \Rightarrow d_1^2 = \frac{-b}{a} d_2^2
	\end{align*}
	Now $ \frac{-b}{a} = -ba = -d(V)$ in $K^{\times}/(K^{\times})^2$ so the above says that $d(V) = - (\frac{d_1}{d_1})^2 = -1$ in the group of square classes. \newline 

	\indent Next we show $2. \to 3.$. 
	Consider any diagonalization of $V$. Its corresponding quadratic form $f = a X_1^2 + b X_2^2$ has $ab = -1$, thus $a = -b$ in the group of square classes. 
	But from \Cref{ex:hyperbolic-plane} we have $f \sim a Y_1 Y_2$, a form which clearly represents any element in $K^{\times}$, in particular 1.
	Now \Cref{prop:rep-criterion1} gives us $V \cong \langle 1 \rangle \perp V'$, and from our previous observation we must have $V' \cong \langle -1 \rangle$ as we wanted. \newline

	\indent $V$ is clearly non-degenerate as its corresponding symmetric matrix is invertible. 
	Now $(x, x)$ is an isotropic vector in $\langle 1, -1 \rangle$ for any $x \in K$ so $V$ is isotropic as well, concluding the proof.
 \end{proof}

 \begin{remark}
	 As we outlined in the proof, note that \Cref{ex:hyperbolic-plane} shows that $\mb{H}$ has corresponding equivalence class of quadratic forms containing the form $X_1 X_2$, perhaps why the name is called hyperbolic.
 \end{remark}
 
 \begin{proposition}\label{prop:isotropic-hyperbolic}
	A nondegenerate quadratic space $(V,B)$ is isotropic iff $V$ contains a hyperbolic plane, in which case $V$ represents all of $K^{\times}$.
\end{proposition}

\begin{proof}
	If $V$ contains a hyperbolic plane, it is isotropic as $\mb{H}$ is isotropic by the previous theorem.
	Conversely, if $(V,B)$ is a nondegenerate quadratic space that is isotropic, we have that a nonzero $v \in V$ such that $B(v,v)$ is zero.
	As $V$ is nondegenerate, there is (by the isomorphism $V \to V^*$ defined by $B$) a nonzero element $z$ such that $B(v,z) = 1$.
	Then consider $y = z - \frac{B(z,z)}{2}v$. 
	It is linearly independent from $v$ as $z$ is (any multiple of $v$ has $B(kv,v) = kB(v,v) = 0$). 
	Furthermore 
	\begin{align*}
		B(y,y) = B(z - \frac{B(z,z)}{2}v,z - \frac{B(z,z)}{2}v) &= B(z,z) - B(z,z) B(v,z) + \frac{B(z,z)^2}{4}B(v,v)\\
																&= B(z,z) - B(z,z)\\
																&= 0
	\end{align*}
and 
\begin{align*}
	B(y,v) = B(z,v) - \frac{B(z,z)}{2}B(v,v) = B(z,v) = 1
\end{align*}
so the quadratic form $f$ corresponding to the $Kv \oplus Ky$ is $2X_1X_2 \sim X_1X_2$ (once again as char $K$ is different from 2) so $V$ contains a hyperbolic plane $\mb{H}$ as claimed. \newline

\indent Finally, to show that $V$ represents all of $K^{\times}$ in the above cases it suffices to show that $\mb{H}$ does. 
But as noted, the quadratic form $X_1X_2$ lying in the equivalence class of forms for $\mb{H}$ clearly does so $\mb{H}$ does as we want.
\end{proof}

Translating this into the language of quadratic forms we obtain the following corollary.

\begin{corollary}
	If $f$ is a nondegenerate quadratic form representing 0, then $f \sim f_1 + g$ where $f_1$ is hyperbolic ($f_1 \sim X_1^2 - X_2^2 \sim X_1 X_2$). 
	Moreover $f$ is isotropic and represents all of $K^{\times}$.
\end{corollary}

\begin{proposition}\label{prop:represent-prop1}
	A nondegenerate quadratic form $Q$ represents $d \in K^{\times}$ iff $Q \perp \langle -d \rangle$ is isotropic.
\end{proposition}
\begin{proof}
	We may assume that $Q = d_1 X_1^2 + \dots  + d_n X_n^2$ in which case recall that $Q \perp \langle -d \rangle \sim d_1 X_1^2 + \dots  + d_n X_n^2 - dZ^2$ for a new variable $Z$.
	If $Q$ represents $d$, so that $Q(x) = d$ for some $x \in K^n$, then 
	\[(Q \perp \langle -d \rangle)(x,1) = Q(x) + \langle -d \rangle(1) = d - d = 0\]
	shows that $Q \perp \langle -d \rangle$ is isotropic. \newline
\indent Conversely, suppose that $Q \perp \langle -d \rangle$ is isotropic, so that $(Q \perp \langle -d \rangle)$ evaluated at $(x_1, \dots, x_n, y) = 0$.
If $y = 0$, then $(x_1, \dots, x_n) \neq 0$ and also $Q(x_1, \dots, x_n) = 0$.
Hence $Q$ is isotropic by our previous proposition and represents all of $K^{\times}$ and $d$ in particular.
Otherwise, $y \neq 0$ and we have 
\[Q(x_1, \dots, x_n) = dy^2 \Rightarrow Q(x_1/y, \dots, x_n/y) = d\]
so $Q$ still represents $d$ as wanted.
\end{proof}

We conclude this section with a fact about quadratic forms we will need to use later in the proof of the Hasse-Minkowski theorem, but for which we do not have the space to give it an adequate context or proof.

\begin{proposition}\label{discriminant-splitting}
	Let $f$ is a four-dimensional quadratic form over a field $K$ of discriminant $d$.
	Then $f$ is isotropic over $K(\sqrt{d})$ if and only if it is isotropic over $K$.
\end{proposition}
\begin{proof}
	See \cite{ATOQF}, (V.3.24) and (VII.3.1)
\end{proof}

\subsection{Witt's Cancellation Theorem} 

\begin{theorem}\label{theo:geometric-witt-cancel}
	If $(V, B)$ and $(V', B')$ are isomorphic and non-degenerate, then every injective metric morphism $s: U \to V'$ from a subvector space $U$ of $V$ can be exteneded to a metric isomorphism of $V$ onto $V'$.
\end{theorem}

\begin{proof}
	We first show we can reduce to the case where $U$ is non-degenerate by showing that $s$ extends to a map $s': U' \to V$, such that $s'|_{U} = s$ and $U'$ contains $U$ and is one-dimension larger.
	In this case, we can keep on extending $s$ to morphisms from larger subspaces until it is either all of $V$ (and we are done) or a nondegenerate subspace. 
	Indeed, if $U$ is degenerate choose a nonzero $x \in \text{rad}(U)$. 
	Consider any linear form $\ell \in U^*$ such that $\ell(x) = 1$.
	By the nondegeneracy of $V$ (hence the natural isomorphism $V \to V^*$) there is a $y$ such that $\ell(u,y) = B(y,u)$ for all $u \in U$.
	In particular $y$ cannot lie in $U$ as it is not orthogonal to $x \in \text{rad}(U)$ so that $U \oplus Ky$ contains $U$ and is one-dimension larger.
Up to replacing $y$ by $y - \frac{B(y,y)}{2}x$ we may also assume that $B(y,y) = 0$. \newline
\indent We may then apply the same construction to the subspace $s(U)$, element $s(x)$ and linear form $\ell' = \ell \circ s^{-1}$ to obtain an element $y'$ and the space $s(U) \oplus Ky'$.
We cali the linear map $\phi: U \oplus Ky \to U' \oplus Ky'$ which is the same as $s$ on $U$ and maps $y$ to $y'$ is our desired one.
It suffices to show it is a metric morphism.
Indeed, for any $u + by \in U \oplus Ky$ we have that 
\begin{align*}
	B'(\phi(u + by), \phi(u + by)) &= B'(\phi(u), \phi(u)) + 2 B'(\phi(u), \phi(by)) + B'(\phi(by), \phi(by))\\
								   &= B(u,u) + 2b B'(\phi(u), \phi(y)) + b^2 B'(\phi(y), \phi)y))\\
								   &= B(u,u)
\end{align*}
from our construction, which shows the reduction. \newline

\indent 
Now assuming that $U$ is nondegenerate, we show our theorem by induction on the dimension of $U$.
As $V \cong V'$, we assume by a little abuse of notation that $V = V'$ and $B = B'$ so that $s$ is an automorphism of $V$.
When $\text{dim}(U) = 1$ and $U$ is nondegenerate is is generated by an anisotropic element $x$.
As $s$ is metric we have $B(x,x) = B(y,y)$, and since 
\[B(x \pm y, x \pm y) = 2B(x,x) \pm 2B(x,y)\]
cannot be 0 for both $\pm 1$ (otherwise $B(x,x) = 0$ one of $x \pm y$ is also anisotropic. 
Up to replacing $y$ with $-y$ we may assume $x + y$ is.
Then the linear map $\sigma: V \to V$ such that
\[\sigma(v) = v - \frac{2B(v, x+y)}{B(x+y, x+y)} (x+y)\]
sends $(x+y)$ to $-(x+y)$ and is the identity on $K(x+y)^{\perp}$.
Now $(y-x) \in K(x+y)^{\perp}$ so 
\[\sigma(x) = \frac{1}{2}\sigma((x+y) + (x-y)) = \frac{1}{2} (-x - y) + (x - y) = -y\]
so $-\sigma$ is a metric automorphism extending $s$. \newline
\indent Now when $\text{dim}(U) > 1$, consider any anisotropic $y \in U$.
By \Cref{prop:nondegen-decomp} we have $U = Ky \perp (Ky)^{\perp}$. 
From the case as above there exists an automorphism $\sigma$ of $V$ extending the map on $Ky$.
Then $\sigma^{-1} s : U \to V$ is the identity on $Ky$ and since it is metric (being the composite of two metric morphisms) on $U$, it must maps $Ky^{\perp}$ onto $V'$, the orthogonal complement of $U$ in $V$. By our inductive hypothesis this extends to a map $\phi': (Ky)^{\perp} \to V$. 
Then the linear map $\varphi$ defined as the identity on $Ky$ and is $\phi'$ on $(Ky)^{\perp}$ extends $\sigma^{-1}s$ so $\varphi \circ \sigma$ is an extension of $s$ as desired.
\end{proof}

Applying the above theorem to the language of quadratic forms we obtain the following corollary.

\begin{corollary}\label{coroll:alg-witt-cancel}
	Let $f = g + h$ and $f' = g' + h'$. If $f \sim f'$ and $g \sim g'$ then $h \sim h'$.
\end{corollary}
\begin{proof}
	If $f \sim f'$ are $n$-ary quadratic forms, then the corresponding quadratic spaces on $K^n$ are isomorphic.
	Then $g \sim g'$ gives an isomorphism from their corresponding spaces, which from the above theorem extends to an isomorphism between the $K^n$.
	Restricting the isomorphism to the subspaces defined by $h$ and $h'$ gives us our result.
\end{proof}


\subsection{Quadratic Forms over Finite Fields}

With the results given on quadratic forms, we believe this is the best time to present the following results regarding those over finite fields $\mb{F}_q$, for $q = p^n$ for some prime $p$.

\begin{proposition}\label{finite-field-isotropic}
	A quadratic form over $\mb{F}_q$ of rank $\geq 2$  (respectively $\geq 3$) represents all elements of $\mb{F}_q^{\times}$ (respectively $\mb{F}_q$).
\end{proposition}
 
\begin{proof}
	If $q = 2^n$, then $\mathbb{F}_q$ is of char 2 and the Frobenius $x^2$ is an isomorphism. 
	In particular quadratic forms of rank $\geq 1$ in this case must always represent all of $\mathbb{F}_q$ so we may assume that $p$ is an odd prime.\newline 

	\indent We first show the statement for quadratic forms of rank $\geq 2$.
	It of course suffices to do it for rank 2 as higher ranks can only represent more values.
	Let $d \in \mathbb{F}_q^{\times}$.
		We will pair off each $x \in \Z/p\Z$ with the residue class $- (d+x) \in \Z/p\Z$ (so that their sum is $-d \bmod{p}$).
The only value paired with itself is then $ \frac{d}{2}$, giving us a total of $ \frac{p+1}{2}$ such pairs.
Now among the $ \frac{p-1}{2}$ quadratic residues in $( \Z /p\Z)^*$, if $\frac{d}{2}$(now an element of $\Z/p\Z$) is among them, say $a^2 = \frac{d}{2}$ then $a^2 + a^2 \equiv \frac{d}{2} +  \frac{d}{2} \equiv d \bmod{p}$.
Otherwise, if $\frac{d}{2}$  is not a quadratic residue, the $ \frac{p-1}{2}$ quadratic residues, along with $0$, are the $ \frac{p-1}{2} + 1$ distinct values in $\Z/p\Z$ which are squares of integers. 
Without the pair consisting of just $\frac{d}{2}$, there are $\frac{p-1}{2}$ pairs, so by pigeonhole there must be one pair $(x,y)$ with both values a square, and the solutions to $a^2 = x, b^2 = y$ then have $a^2 + b^2 \equiv d \bmod{p}$ by construction. \newline

The statement for quadratic forms of rank $\geq 3$ once again reduces to just the case for forms of rank $3$.
From our previous result is suffices to show that all quadratic forms $f$ over $\mb{F}_q$ are isotropic.
If it is identically zero this is clear otherwise  there is a $x \in K^3$ such that $f(x) = d \neq 0$.
But then from \Cref{coroll:rep-criterion1} we have $f \sim f' + dZ^2$, where $f'$ is of rank 2. 
But we know $f'$ represents $-d$, so $f$ must be isotropic as claimed.
\end{proof}

So quadratic forms of rank 2 over $\mathbb{F}_q$ always represent all of $K^{\times}$, such fields with this property have a nice characterization of their quadratic forms as we show.

\begin{theorem}
	Let $K$ be a field where all quadratic forms of rank 2 represent $K^{\times}$.
	Then every quadratic form of rank $n$ has $|K^{\times}/(K^{\times})^2|$ equivalence classes corresponding to one of the forms
	\[X_1^2 + \dots X_{n-1}^2 + d X_n^2\]
for $d$ representatives of $K^{\times}/(K^{\times})^2$. 
In particular, the rank and discriminant uniquely determine the equivalence class of the quadratic form.
\end{theorem}

\begin{proof}
This is clear if $n = 1$.
For $n \geq 2$ our previous proposition shows it represents $1$ so by \Cref{coroll:rep-criterion1} it is equivalent to $X_1^2 + g$ for $g$ of rank $n-1$, and our inductive hypothesis proves the statement.
The representative class $d$ in $K^{\times}/(K^{\times})^2$ is then the discriminant of our form, and so uniquely deterimines it.
\end{proof}

\begin{corollary}
	There are exactly two equivalence classes of rank $n$ quadratic forms over a finite field $\mb{F}_q$.
\end{corollary}
\begin{proof}
	This follows as $|\mathbb{F}_q^{\times}/ (\mathbb{F}_q^{\times})^2| = 2$ as its multiplicative group is cyclic.
\end{proof}



\section{Quaternion Algebras}


There are analogous constructions of the usual quaternions (over the reals) to an arbitrary field $K$ still of characteristic different from 2.

\begin{definition}[Quaternion Algebra]
	Let $a,b \in K$. We define the quaternion algebra $\qa{a}{b}$ as the $K$-algebra with two generators $i,j$ defined by the relations
\[i^2 = a, \quad j^2 = b, \quad \text{ and } ij - ji\]
\end{definition}

letting $k = ij$ we see that $\qa{a}{b}$ is 4-dimensional over $K$ with a basis $\{1,i,j,k\}$, where
\[k^2 = ijij = -i^2j^2 = -ab\]
Also note that any two elements among $\{i,j,k\}$ anti-commute (i.e. $ik = -ki$ and $jk = -jk$).
In this way, the usual quaternions over $\R$ is $(\frac{-1, -1}{\R})$.
We now record some basic facts regarding quaternion algebras.

\begin{proposition}\label{prop:basic-facts}
	We have
\begin{enumerate}
	\item$( \frac{a,b}{K})  = ( \frac{ax^2,by^2}{K}), \quad a,b,x,y \in K^{\times}  $
	\item $( \frac{-1,1}{K}) \cong \mb{M}_2(K)$ (The $2 \times 2$ matrices over $K$) 
	\item The center of $( \frac{a,b}{K})$  is $K$ (the $K$-span of $1$).
	\item $( \frac{a,b}{K})$ is a simple algebra (no two-sided proper ideals).
\end{enumerate}
\end{proposition}

\begin{proof}
	For 1.), note that $\qa{ax^2}{by^2}$ has basis $\{1, i', j', k'\}$ satisfying $i'^2 = ax^2, j'^2 = by^2$.
	However, since $xi, yj \in \qa{a}{b}$ satisfy $(xi)^2 = ax^2, (yj)^2 = by^2$ and $(xi)(yj) = (-yj)(xi)$ (as scalars commute) the map $\qa{a}{b} \to \qa{ax^2}{by^2}$ sending $xi \to i'$ and $yj \to j'$ is an isomorphism (with inverse sending $i' \to i/x, y' \to y/j$). \newline
	\indent For 2.) we set $1 = \begin{bmatrix} 1 & 0 \\ 0 & 1 \end{bmatrix}, i = \begin{bmatrix} 0 & 1 \\ -1 & 0 \end{bmatrix}, j = \begin{bmatrix} 0 & 1 \\ 1 & 0 \end{bmatrix}$ and it is straightforward to verify that it satifies the relations of the quaternion algebra. 
	It suffices then to show that $1,i,j$ and $k = \begin{bmatrix} 1 & 0 \\ 0 & -1 \end{bmatrix}$ span $\mb{M}_2(K)$ as a $K$-vector space in which case the isomorphism is given on generators as above.
	Indeed, we have $\begin{bmatrix} a & b \\ c & d \end{bmatrix} = \frac{a+d}{2}1 + \frac{b -c}{2}i + \frac{b+c}{2}j + \frac{a -d }{2}k$ shows surjectivity, and  if $x1 + yi + zj + wk = \begin{bmatrix} x+w & z+y \\ z-y & x-w \end{bmatrix} = \begin{bmatrix} 0 & 0 \\ 0 & 0 \end{bmatrix}$, then as $K$ is of characteristic not 2 the system of equations gives us $x = y = z = w = 0$ showing injectivity.\newline
	\indent For 3. and 4. let $\overline{K}$ be the algebraic closure of $K$ and note that $\overline{K} \otimes_K \qa{a}{b}$ = $(\frac{a,b}{\overline{K}})$.
	Every element is a square in $\overline{K}$, so from both results 1. and 2. we conclude that $(\frac{a,b}{\overline{K}}) \cong \mb{M}_2(\overline{K})$.
	We know the center $Z(\mb{M}_2(\overline{K}))$ are just the scalars $\overline{K} \cdot 1$.
	Any element in its center must lie in the center $\qa{a}{b} \subset \mb{M}_2(\overline{K})$. Conversely, an element is in $Z(\qa{a}{b})$ if and only if it commutes with $\{1,i,j,k\}$ which is independent of the base field, so  $Z(\qa{a}{b})  = Z(\mb{M}_2(\overline{K})) \ic \qa{a}{b}$ so must also be a scalar in $K$ showing 3. \newline
	\indent If $I$ were a proper two-sided ideal of $\qa{a}{b}$, then it is also a $K$-vector space of dimension less than 4. 
	Then $I \otimes_K \overline{K}$ is also a two-sided ideal of $\mb{M}_2(\overline{K}))$ which is proper as it has the same dimension. 
	As $\mb{M}_2(\overline{K}))$ is a simple algebra this is impossible, so $\qa{a}{b}$ is also simple.
\end{proof}


\begin{definition}[Pure Quaternions]
	A quaternion $\alpha + \beta i + cj + dk \in A = \qa{a}{b}$ is called a \textit{pure quaternion} if $\alpha = 0$.
	The set of pure quaternions in $A = \qa{a}{b}$ will be denoted as $A_0$.
\end{definition}

We show there is a "basis free" interpretation of pure quaternions. 

\begin{proposition}
	A nonzero $v$ is a pure quaternion if and only if $v \notin K$ and $v^2 \in K$.
\end{proposition}

\begin{proof}
	Direct calculation shows that if $v = \alpha + \beta i + cj + dk$, then 
	\begin{align*}
		v^2 = (\alpha^2 + a \beta^2 + b c^2 -abd^2) + 2a(\beta i + cj + dk)
	\end{align*}
If $v$ is a nonzero pure quaternion then $v \notin K$, then $\alpha = 0$ so the above must be a scalar in $K$.
Conversely, if $v \notin K$ as above, one of $\beta, c ,d$ must be nonzero, so for the above square to be 0 we must have $2\alpha = 0$, so $\alpha = 0$ (once again as char $K$ different from 2) so $v$ is a nonzero pure-quaternion.
\end{proof}

\begin{corollary}\label{coroll:pure-quaternions}
	If $A = \qa{a}{b}$ and $A' = \qa{a'}{b'}$, then any $K$-algebra isomorphism $\phi: A \to A'$ restricts to an isomorphism $A_0 \to A'_0$ between the pure quaternions.
\end{corollary}

\begin{proof}
	Given the purely algebraic characterization from the proceeding proposition, $\phi$ must map $A_0$ into $A'_0$.
	However, as $A$ is a simple algebra by \Cref{prop:basic-facts}, 4. $\phi$ must be injective.
	As both $A_0$ and $A'_0$ are three-dimensional over $K$ we conclude the map must also be surjective.
\end{proof}

\subsection{Quaternions as Quadratic Spaces}
We now make quaternions into a quadratic space.
For a quaternion $x = \alpha + \beta i + cd + jk$ we define its conjugate $\overline{x} = \alpha - \beta i - cd - jk$.
The conjugation operation is "almost" an algebra automorphism in that it commutes with addition and scaling by $K$, but in general reverses order of multiplication in that $\overline{(xy)} = \overline{y}\, \overline{x}$.
Note for any pure quaternion $y$ that $\overline{y} = -y$.

\begin{definition}[Norm and Trace]
	For $x \in \qa{a}{b}$, we define the \textit{norm} of $x$ as $x \overline{x}$ and the \textit{trace} of $x$ as $x + \overline{x}$.  
\end{definition}

We will define the quadratic space structure on $(\qa{a}{b}, B)$ by giving it a symmetric bilinear form, which explicitly is $B(x,y) = \frac{1}{2}(x \overline{y} + y \overline{x})$. 
As
\[\overline{B(x,y)} = \overline{\frac{1}{2}(x \overline{y} + y \overline{x})} = \frac{1}{2}(\overline{x} y + \overline{y} x) = B(x,y)\]
it is preserved by conjugation and so indeed takes values in the field $K$.
Bilinearity and symmetry are similarly straightforward checks, so this is indeed a well-defined form.

\begin{proposition}
	The quadratic space $(\qa{a}{b}, B)$ has an orthogonal basis $\{1, i, j, k\}$ and is isomorphic to $\langle 1, -a, -b, ab \rangle$.
\end{proposition}

\begin{proof}
	For any pure quaternion 
	For any $y \in \{i,j,k\}$ we have $B(1,y) = \frac{1}{2}(y - y) = 0$.
	And if $w,z$ are any two pure quaternions we have, in general, that 
	\[B(w,z) = \frac{1}{2}(w \overline{z} + z \overline{w}) = -\frac{1}{2}(w z + z w)  \]
	so $w,z$ are orthogonal if and only if they anti-commute, hence $i,j,k$ are also mutually orthogonal hence $\{1,i,j,k\}$ is an orthogonal basis for $B$. 
	This orthogonal basis then gives us the desired isomorphism as we have 
	\[B(1,1) = 1, \quad B(i,i) = -i^2 = -a, \quad\quad B(j,j) = -j^2 = -b, \quad \quad B(k,k) = -k^2 = ab\]
\end{proof}

\begin{remark}
	From the diagonalization above, if $x = \alpha + \beta i + cd + dk \in \qa{a}{b}$  then $N(x) = \alpha^2 - a\beta^2 - b c^2 + d^2ab$.
\end{remark}

As is the case with with the usual quaternions over the reals, we see that the norm is multiplicative, and hence gives an useful condition for checking invertibility of elements. 

\begin{proposition}\label{prop:invertibility}
	Consider any $x,y \in \qa{a}{b}$. Then  
	\begin{enumerate}
		\item $N(xy) = N(x)N(y)$ 
		\item $x$ is invertible if and only if $N(x) \neq 0$
	\end{enumerate}
\end{proposition}
\begin{proof}
	We straightforwardly verify, noting that $y \overline{y}$ is a scalar in $K$ and so lie in the center, that 
	\[N(xy) = (xy) \overline{(xy)} = x(y \overline{y}) \, \overline{x} =  (y \overline{y}) (x \overline{x}) = N(x)N(y)\]
	Now if $x$ is invertible, say $xy = 1$ we have $1 = N(1) = N(xy) = N(x)N(y)$ so $N(x) \neq 0$.
	Conversely, if $N(x) = k$, then $x^{-1} = \frac{\overline{x}}{k}$ since $x \frac{\overline{x}}{k} = \frac{x \overline{x}}{k} = \frac{N(x)}{k} = 1$.
\end{proof}

We now show that algebraic equivalence of quaternionic algebras is equivalent to the more geometric notion of being isomorphic as a quadratic space, more or less a consequence of the quadratic map $N(-)$ being defined algebraically.

\begin{proposition}
	Let $A = \qa{a}{b}$ and $A' = \qa{a'}{b'}$, then the following are equivalent.
	\begin{enumerate}
		\item $A$ and $A'$ are isomorphic as $K$-algebras.
		\item $A$ and $A'$ are isomorphic as quadratic spaces.
	\end{enumerate}
\end{proposition}

\begin{proof}
	First suppose there is an isomorphism $\phi:A \to A'$ of $K$-algebras.
	We first show that the isomorphism commutes with conjugation, so that for any $x \in A$ that $\phi(\overline{x}) = \overline{\phi(x)}$.
	Letting $x = \alpha + \beta$, with $\alpha \in K$ and $\beta$ a pure quaternion, we have 
	\begin{align*}
		\phi(\overline{x}) = \phi(\alpha -\beta) = \phi(\alpha) - \phi(\beta) = \alpha - \phi(\beta)
	\end{align*}
The last equality as $\phi$ is a $K$-algebra morphism. 
Then note that \Cref{coroll:pure-quaternions} gives us $\phi(\beta) \in A'_0$ so 
\[\alpha - \phi(\beta) = \overline{a + \phi(\beta)} = \overline{\phi(x)}\]
as claimed. 
Now $\phi$ then defines an isomorphism of quadratic spaces as 
\[B'(x,x) = N(\phi(x)) = \phi(x) \overline{\phi(x)} = \phi(x) \phi(\overline{x}) = \phi(x \overline{x}) = \phi(N(x)) = N(x) = B(x,x)\]

Conversely, assume that $A$ and $A'$ are isomorphic as quadratic spaces. 
Then by \Cref{coroll:alg-witt-cancel} we have that an isomorphism $\varphi: A_0 \to A'_0$ as quadratic spaces. 
Then note that 
\[\varphi(i)^2 = -B'(\varphi(i), \varphi(i)) = -B(i, i) = a, \quad \varphi(j)^2 = -B'(\varphi(j), \varphi(j)) = -B(j, j) = b\]
Orthogonality between $\varphi(i)$ and $\varphi(j)$ then gives us that
\[0 = B'(\varphi(i). \varphi(j)) = \frac{1}{2}(\varphi(i) \varphi(j) + \varphi(j) \varphi(i)) \Rightarrow \varphi(i)\varphi(j) = - \varphi(j) \varphi(i)\]
By the same reasoning one verifies $\varphi(ij)$ is orthogonal to $\varphi(i), \varphi(j)$ so the three span $A'_0$
therefore $\varphi(i)$ and $\varphi(j)$ generate $A_0$ over $K$, and satisfy the same relations defiining $\qa{a}{b}$, hence they are isomorphic.
\end{proof}

\subsection{The Hilbert Symbol}

We now define the Hilbert symbol, and give an equivalence of its equality with what will be called the  splitting of a quaternion algebra.

\begin{definition}[Hilbert Symbol]
	For any field $F$, and $a,b \in K^{\times}$ we define the \textit{Hilbert Symbol} of $a$ and $b$ (relative to $F$) as 
\[(a,b) =
\left\{
	\begin{array}{ll}
		1  & \mbox{if $Z^2 -aX_1^2 -bX_2^2 = 0$ has a nonzero solution in $K^3$} \\
		-1 & \mbox{otherwise }
	\end{array}
\right.\]
\end{definition}

\begin{remark}
	Since $a,b$ lie in a field the Hilbert symbol $(a,b)$ doesn't change when multiplying either $a,b$ by squares in $K^{\times}$, so this is well-defined on $K^{\times}/(K^{\times})^2$ the group of square classes.
	In addition, note that by \Cref{prop:represent-prop1} that $(a,b) = 1$ is equivalent to $\langle a, b \rangle$ representing 1.
\end{remark}

\begin{proposition}\label{prop:hilbert-norm}
	Let $a,b \in K^{\times}$ and let $K_b = K(\sqrt{b})$.
	Then $(a,b) = 1$ if and only if $a$ belongs to the group $N K_b^{\times}$ of norms of elements in $K_b^{\times}$.
\end{proposition}

\begin{proof}
	If $b = c^2$ is a square in $F$ then $(c,0,1)$ is a solution to $Z^2 -aX_1^2 -bX_2^2 = 0$  so $(a,b) = 1$.
	Since $K_b = 1$ we have $N K_B^{\times} = N K^{\times} = K^{\times}$ which contains $a$, showing this case. \newline
	\indent Otherwise $b$ is not a square in $F$, in which case $K_b = K(\sqrt{b})$ is a quadratic extension.
	Every element in $K_b$ can be expressed as $x + y\sqrt{b}, x,y \in K$ with $N(x + y\sqrt{b}) = x^2 - by^2$.
	So if $a$ lies in $N K_b^{\times}$ we have $a = x^2 - by^2$ then $(1,1,y)$ is a solution to $Z^2 -aX_1^2 -bX_2^2 = 0$  so $(a,b) = 1$.
	Conversely, if $(a,b) = 1$ then $X_1 \neq 0$ otherwise $b$ would be a square. 
	Then we have a solution $(x_1,x_2,z)$ to $Z^2 -aX_1^2 -bX_2^2 = 0$, in which case $a = (Z/X_1)^2 - b(X_2/X_1)^2 = N(Z/X_1 + (X_2/X_1)\sqrt{y})$ lies in $N K_b^{\times}$ as claimed. 
	\end{proof}


\begin{theorem}\label{theo:splitting}
	The following conditions for $A = \qa{a}{b}$ are equivalent, in which case we say that $A$ \textbf{splits} over $K$ 	
	\begin{enumerate}
		\item $A \cong \qa{-1}{1}$.
		\item $A$ is not a division algebra.
		\item $A$ is isotropic as a quadratic space.
		\item $A$ is hyperbolic as a quadratic space. ($A \cong \mb{H} \perp \mb{H}$)
		\item We have $(a,b) = 1$ 
	\end{enumerate}
	
\end{theorem}

\begin{proof}
	$1. \to 2.$, as if $A \cong \qa{-1}{1}$ we know from $\Cref{prop:basic-facts}$ that $A \cong \mb{M}_2(K)$ is not a divison algebra as it has zero-divisors. \newline
	\indent $2. \to 3.$ as non-invertible element in $A$ must represent 0 by \Cref{prop:invertibility},\newline
	\indent To show $3. \to 4.$ note that as $A$ is isotropic then $A$ contains $\mb{H}$ ( by \Cref{prop:isotropic-hyperbolic}).
	But as $A$ is regular we know that $A \cong \mb{H} \perp \mb{H}^{\perp}$ (by \Cref{prop:nondegen-decomp}). 
	However, note that on discriminants we have
	\[d(A) = (-a)(-b)ab = 1 = d(\mb{H})d(\mb{H}^{\perp}) = -d(\mb{H}^{\perp})\]
	implies that $d(\mb{H}^{\perp}) = -1$, hence $\mb{H}^{\perp} \cong \mb{H}$ (by \Cref{theo:hyperbolic-plane}) establishing the isomorphism. \newline
	\indent To prove $4. \to 5.$ recall again (by \Cref{theo:hyperbolic-plane}) that for any $k \in K^{\times}$ we have $\langle k, -k \rangle \cong  \langle 1, -1 \rangle = \mb{H}$.
	Then 
	\begin{align*}
		\langle1, a, b, -ab \rangle \cong \mb{H} \perp \mb{H} = \langle 1, -1, 1, -1\rangle &\Rightarrow \langle1, a, -a, b, -b, -ab \rangle \cong \langle 1, -1, 1, -1, a, b\rangle\\
															  &\Rightarrow \langle1, 1, -1, 1, -1, -ab \rangle \cong \langle 1, -1, 1, -1, a, b\rangle\\ 
	&\Rightarrow \langle1, -ab \rangle \cong \langle a, b\rangle
	\end{align*}
	the last implication from \Cref{coroll:alg-witt-cancel}. 
	Now $\langle 1, -ab \rangle$ clearly represents 1 so $\langle a, b \rangle$ does as well.
	As remarked, this is equivalent to $(a,b) = 1$.\newline
	\indent We show $5. \to 2.$. 
	From \Cref{prop:hilbert-norm} we know that $a$ lies in $N K_b^{\times}$.
	We once again distinguish two cases. 
	If $b = c^2$ is a square in $K$, then $(c - j)(c+j) = 0$ gives us that $A$ has zero-divisors so isn't a division algebra.
	If $b$ is not a square in $K$, we once again have $z^2 -a - by^2 = 0$ for nonzero $z,y$, but this is precisely $N(z + i + yj)$ hence $(z + i + yj)$ is not invertible. (Again by \Cref{prop:invertibility}).\newline
	\indent To show that $2. \to 1.$ we have to use the Wedderburn-Artin Theorem, which gives us that since $A$ is a simple finite-dimensional $K$-algebra (from \Cref{prop:basic-facts}), it is a matrix algebra $\mb{M}_n(D)$ for over a division ring over $K$.
	However, since $A$ is not a division algebra $m \geq 2$, but since $\text{dim}_K(A) = 4$ we must have $m = 2$ and $D = K$ finishing the proof.
\end{proof}



\section{The Hasse-Minkowski Principle}



Here $K$ will always denote a finite extension of $\Q_p$ (in our case we are interested in the completion of a number field at a nonzero prime). 
We will fix notation for the rest of this section here.
$\mc{O}_K$ willdenote its corresponding discrete valuation ring with maximal ideal $\mk{m}$ and residue field $k$, and $\pi$ any uniformizer.
\newline 

	\indent If $f^{(i)} \in \mc{O}_K[X_1, \dots, X_n]$ are are all homogeneous polynomials (say a quadratic form) , note any nonzero solutions in $(\mc{O}_K)^m$ defined up to a scalar. 
	So if $(x_1, \dots, x_m)$ is a nonzero solution in $K^m$, then choosing an element $y$ such that $v(y) = \text{min}(v(x_1), \dots, v(x_m))$, we have that $(y^{-1} x_1, \dots, y^{-1} x_m)$ is another solution (now guaranteed to be in $\mc{O}_K^m$) such that one coefficient is invertible in $\mc{O}_K$. 
	We will call such a solution \textit{primitive} (following \cite{ACIA}), similarly defined in $\mc{O}_K/\m^n$.





\begin{lemma}\label{lemm:hensel}
	Let $f \in \mc{O}_K[X]$ and $f'$ its derivative.
	If $x \in \mc{O}_K$, with $n,k \in \Z$ satisfying $0 \leq 2k < n$ and $f(x) \equiv 0 \pmod{\m^n}, v(f'(x)) \equiv k$.
	Then one can always find $y \in \mc{O}_K$ satisfying 
	\[f(y) \equiv 0 \pmod{\m^{n+1}},\quad v(f'(y)) \equiv k, \quad y = x \pmod{\m^{n-k}}\]
\end{lemma}

\begin{proof}
	This proof uses the same idea as the proof of Hensel's lemma, and we sketch it here. 
	If $\pi$ is an uniformizer for $K$, by our assumptions we may write $f(x) = \pi^n a, f'(x) = \pi^k u$, with $a, u \in \mc{O}_K$ and $u$ invertible.
	Since $y \equiv x \pmod{p^{n-k}}$ we suspect $y$ to be of the form $x + p^{n-k}z$ or $z \in \mc{O}_K$.
	After expanding out, we have for some $b \in \mc{O}_K$ that 
\begin{align*}
	f(y) = f(x) + p^{n-k}z f'(x) + p^{2n - 2k} b =    p^n(a + uz) +  p^{2n - 2k} b
\end{align*}
We may choose $z$ such that $a + uz \equiv 0 \pmod{\m}$, and since $2n - 2k > n$ we have $f(y) \equiv 0 \pmod{\m^{n+1}}$.
Expanding the same way for $f'(y)$ shows that $f'(y) \equiv p^k u \pmod{p^{n-k}}$ and so $v(f'(y)) \equiv k$ still, as we wanted.
\end{proof}

This allows us to lift solutions mod $\m^n$ in many ways as we describe using the following theorem.

\begin{theorem}\label{theo:hensel}
	Let $f \in \mc{O}_K[X_1, \dots, X_m]$, $x = (x_i) \in (\mc{O}_K)^m$	with $n,k \in \Z$ satisfying $0 \leq 2k < n$ and $0 \leq j \leq m$ an index such that
	\[f(x) \equiv 0\pmod{\m^{n}},\quad v( \frac{\partial f}{ \partial X_j}(x)) = k\]
	Then $f$ has a zero $y \in (\mc{O}_K)^m$ congruent to $x \pmod{\m^{n-k}}$
\end{theorem}
\begin{proof}
	If $m = 1$ we easily reduce to \Cref{lemm:hensel}. 
	Letting $x_0 = x$, we obtain an $x_1$ from $x$ and in general a $x_i, i > 1$ such that $f(x_i) \equiv 0 \pmod{\m^{n+i}}$ amd $y = x \pmod{\m^{n-k + i}}$.
	This defines a Cauchy sequence $x_i$ whose limit is a zero satisfying our required conditions. \newline
	\indent The case when  $m > 1$  reduces to the first case, as letting $\overline{f} \in \mc{O}_K[X_j]$ be $f$ with $X_i$ substituted with $x_i$ for $i \neq j$, satisfies our above conditions for some $y_j \equiv x_j \pmod{\m^{n-k}}$.
	Then $y$, equal to $(x_i)$ but with $x_j$ switched with $y_j$ is a zero satisfying our requirements as wanted.
\end{proof}

This theorem gives us some existence theorems for zeros in many special cases of interest to us. 

\begin{corollary}\label{cor:simple-zero-lift}
	A simple zero $\overline{x}$ of $f^{(i)} \pmod{\m}$ always lifts to a zero $x$ in $(\mc{O}_K)^m$.
	(Here simple implies one of the partial derivates is nonzero at $\overline{x}$).
\end{corollary}
 
\begin{proof}
	Indeed, lifting the $\overline{x}_i$ to any $x_i \in \mc{O}_K$ such that $\overline{x}_i  = x_i \pmod{\m}$, then $x = (x_i)$ satisfies the condition of \Cref{theo:hensel} for $n=1, k=0$, as the $j$ for which the partial at $\overline{x}_j \neq 0$ then satisfies $v(x_j) = 0$.
\end{proof}

\subsection{Quadratic Forms Over Number Fields}

In the case of quadratic forms $f$ the lifting of solutions depends on the characteristic of the residue field $k$.

\begin{proposition}\label{prop:char-not-2-lift}
	Suppose char $k \neq 2$. If $f(x)$ is a quadratic form and $b \in \mc{O}_K$, we have that any primitive solution $x$ such that of $f(x) - b \equiv 0 \bmod{\m}$ lifts to one in $(\mc{O}_K)^m$ if $x$ does not annihilate all of $\frac{\partial f}{\partial X_j} $ modulo $\m$. 
	This condition is fulfilled if $\text{det}(a_{ij})$ is invertible.
\end{proposition}

\begin{proof}
	Note that $ \frac{\partial f}{\partial X_j} = 2 \sum_i a_{ij} X_j$ is the twice the $j$th entry of matrix product $(a_{ij}) x$. 
	By our assumption that $x$ doesn't annihilated every partial, at least one entry must be nonzero modulo $\m$ and we are done by \Cref{cor:simple-zero-lift}.
	This condition is satisfied when $\text{det}(a_{ij})$ as $x$ is nonzero being primitive, so $(a_{ij}) x \neq 0$ modulo $\m$. 
\end{proof}


When char $k = 2$ the previous argument doesn't work as $2 = 0$.
The best alternative is considering solutions mod $\m^n$ for sufficient $n$ such that $\mc{O}_K /\m^n$ is not of characteristic 2.
The smallest such $n$ will depend on the ramification of $\mc{O}_K$ over $\Q_2$.

\begin{proposition}
	Suppose char $k = 2$ and denote $e$ the ramification index of $K/\Q_2$. 
	If $Q(x)$ is a quadratic form and $b \in \mc{O}_K$, we have that any primitive solution $x$ such that of $Q(x) - b \equiv 0 \bmod{\m^{2e + 1}}$ lifts to one in $(\mc{O}_K)^m$ if $x$ does not annihilate all of $\frac{\partial Q}{\partial X_j} $ modulo $\m^{e+1}$. 
	This condition is similarly fulfilled if $\text{det}(a_{ij})$ is invertible.

\end{proposition}

\begin{proof}
	The proof is analogous to \Cref{prop:char-not-2-lift}, noting that $e+1$ is the smallest value for which $2 \neq 0$ in $\mc{O}_k/\m^{e+1}$.
	We thus require non-zero partials modulo $\m^{e+1}$, whose valuation is at most $e$.
	As in the notation of \Cref{theo:hensel}, we thus have $k \leq n$ and requiring $0 \leq 2k < n$, we see $n = 2e + 1$ is the smallest choice for which we may always lift a solution to $(\mc{O}_K)^m$. 
\end{proof}




In this section we establish just enough facts about quadratic forms over number fields to give a proof of the Hasse-Minkowski Theorem afterwards.

\begin{corollary}\label{lifting-up}
	Suppose $\text{char}(k) \neq 2$.
	Any quadratic form $f = d_1 X_1^2 + \dots d_n X_n^2$ such that $d_1, \dots, d_n$ are units in $K$ and $n \geq 3$ is isotropic.
\end{corollary}
\begin{proof}
	Let $\overline{d}_i = d_i \pmod{\mk{m}}$ for $1 \leq i \leq n$.
	Then $d_i \in U$ implies $\overline{d}_i \neq 0$
	By \Cref{prop:char-not-2-lift} it suffices to show that the rank $n$ quadratic form $\overline{f} = \overline{d}_1 X_1^2 + \dots \overline{d}_n X_n^2$ over $k$ is isotropic for $n \geq 3$, which follows from our results on quadratic forms of finie fields. (\Cref{finite-field-isotropic}).
\end{proof}


\begin{theorem}
	Once again let $\pi$ denote an uniformizer of $K$. 
	For any $\alpha \in \mc{O}_K$, there exists a $\beta \in \mc{O}_K$ such that $1 + 4 \pi \alpha = (1 + 2 \pi \beta)^2$.
	Furthermore, $1 + 4 \mk{m} \subset U^2$, for $U$ the group of units of $K$.
\end{theorem}
\begin{proof}
	We may apply \Cref{theo:hensel} to the polynomial $X^2 - (1 + 4\pi \alpha)$ with $x = 1, k = v(2), n = v(4 \pi \alpha)$ as if $\text{char}(k) \neq 2$ then $k = 0$, otherwise if $v(2) = e > 0$, then $v(4 \pi \alpha) > 2e$ (so we always have $0 \leq 2k < n$).
	Then there exists a square root $c$ of $1 + 4 \pi \alpha$ such that $c \equiv 1 \pmod{2 \pi \alpha}$ and $v(2c) = v(f'(1)) = v(2)$.
	The former equality says that $c = 1 + 2 \pi y$ for some $y \in \mc{O}_K$ as we wanted.
	Finally, the latter equality says $v(c) = 0$ so $c$ is a unit, which applying the above to any $k \in \mk{m}$ gives us $1 + 4 \mk{m} \subset U^2$.
\end{proof}

\begin{corollary}
	If $\text{char} (k) \neq 2$, then $x \in \mc{O}_K$ is a square if and only if $x \pmod {m}$ is a square in $k$.
\end{corollary}

\begin{proof}
	If $x = a^2$ is a square, reducing mod $\mk{m}$ shows $x \pmod{m}$ must be a square in $k$.
	Conversely, let $u = x \pmod{m}$ and suppose it is a square in $k$, i.e. $u = c^2$.
	By dividing $x$ by the Teichmuller representative which is $u$ mod $\mk{m}$ (the square of the one which is $c$ mod $\mk{m}$) it reduces to the case where $x \equiv 1 \pmod {\mk{m}}$.
	As $\text{char}(k) \neq 2$ we have 4 is invertible in $k$, so every such $x$ can be expressed as $1 = 4 \pi a$, $a \in \mc{O}_K$ and our results follows from the previous theorem. 

\end{proof}

\begin{corollary}\label{open}	
	$(K^{\times})^2$ is an open subgroup of $K^{\times}$.
	More specifically, if $c \equiv d \bmod{4 \pi}$ then $c \in d (K^{\times})^2$
\end{corollary}
\begin{proof}
	This says $c/d \equiv 1 \bmod{4 \pi}$, now apply the previous theorem.
\end{proof}


\subsection{Hasse-Minkowski Theorem}

\begin{notation}
	Throughout this subsection, $F \supset \Q$ will denote a number field, and $V$ thet set of equivalence classes of norms (both archimedean and non-archimedean) on $F$.
\end{notation}


Our proof of the Hasse-Minkowksi Theorem over number fields won't be entirely self-contained (our proof will follow \cite{ATOQF} and we will need two major results which we state below without proof.

\begin{theorem}\label{global-square-theorem}
	Let $a \in F^{\times}$.
	Then $a \in (F^{\times})^2$ if and only if $a \in (F_v^{\times})^2$ for each $v \in V$.
\end{theorem}


\begin{theorem}\label{local-global-quat-splitting}
	Let $A$ be a quaternion algebra over $F$. Then $A$ splits over $F$ if and only if $A$ splits over $F_v$ for each $v \in V$.
\end{theorem}

The second theorem may be viewed as a special case of the Albert-Brauer-Hasse-Noether theorem, which is similarly stated but for central simple algebras over $F$.
We now proceed with the proof of our main theorem of this paper.

\begin{theorem}[Hasse-Minkowski Theorem]
	A quadratic form $f$ over $F$ is isotropic if and only if $f$ is isotropic over $F_v$ for each $v \in V$.
\end{theorem}

\begin{proof}
	From the embeddings $F \hookrightarrow F_v$ for each $v \in V$, it is immediate that if $F$ is isotropic, the same element over $F$ shows it is isotropic over the $F_v$. \newline

	\indent To show the conditions are sufficient, we argue depending on the rank $n$ of $f$.
	We may always represent an  $n$-ary form $f$ as $X_1^2 + a_2X_2^2 + \dots + a_n X_n^2$ (corresponding to the quadratic space $\langle 1, a_2, \dots, a_n \rangle$) as $f = a_1X_1^2 + \dots a_n X_n^2$ is a diagonal form, $f$ isotropic if and only if $a_1 f$ is isotropic, and the latter is equivalent to $X_1^2 + a_2X_2^2 + \dots + a_n X_n^2$.
	Rank $1$ forms are never isotropic, so the case $n = 1$ is trivial. \newline

	\indent $n = 2$: Let $f = X_1^2 + aX_2^2$ correspond to the quadratic space $\langle 1, a \rangle$.
Over each $F_v$ if $(x_1, x_2)$ represents then $x_1, x_2 \neq 0$ so $-a = (x_1/x_2)^2$ shows that $-a$ is a square in each $F_v$.
Then $-a$ is a square in $F$ by \Cref{global-square-theorem}.
So if $-a = b^2$ in $F$ we see $(b,1)$ shows $f$ is indeed isotropic over $F$. \newline

\indent $n = 3:$ Letting $f = X_1^2 + a X_2^2 + b X_3^2$ correspond to the quadratic space $\langle 1,a,b \rangle$.
For any $v \in V$, as $\langle 1,a,b \rangle$ is isotropic over $F_v$ we have $\langle -a,-b\rangle$ represents 1 (by \Cref{prop:represent-prop1}).
Then the quaternion algebra $( \frac{-a, -b}{F_v})$ splits (by \Cref{theo:splitting}).
However, then $( \frac{-a, -b}{F})$ splits (by \Cref{local-global-quat-splitting}) so again we equivalently have that $\langle -a, -b \rangle$ represents 1 over $F$, so $\langle 1,a,b \rangle$ is indeed isotropic over $F$. \newline

\indent $n = 4:$   Letting $f = X_1^2 - a X_2^2 - b X_3^2 + c X_4^2$ correspond to the quadratic space $\langle 1,-a,-b,c \rangle$, which has discriminant $abc$.
It suffices to show $f$ is isotropic over $L = F(\sqrt{abc})$ by \Cref{discriminant-splitting}.
Letting $V_L$ denoting the set of equivalence class of norms for $L$, we have the completion $L_v$ contains $\sqrt{abc}$ for any $v \in V_L$, in particular $abc \in L^{\times}/(L^{\times})^2$ so that $\langle 1,-a,-b,c \rangle \cong \langle 1,-a,-b,ab \rangle$.
This is the quadratic form associated to $(\frac{a,b}{L})$, and it represents 0 (as $L_v$ is either $\R, \C$ or is an extension of some $F_v$ where we know it is isotropic) so that $(\frac{a,b}{L})$ splits (by \Cref{theo:splitting} again).
Then $\langle 1,-a,-b,ab \rangle \cong \langle 1,-a, -b, c\rangle$ splits over the global field $ L$ which from the same theorem on splitting says $\langle 1,-a, -b, c\rangle$ is indeed isotropic. \newline 


\indent $n = 5:$ Let $f = g - h$, where $g = a_1 X_1^2 + a_2 X_2^2$ and $h = a_3 X_3^2 + \dots a_nX_n^2$.
Then there is a finite subset $S \subset V$ such that $h$ is isotropic on $F_j$ for $j \in V - S$ by \Cref{lifting-up} (we can take $S$ to be the equivalence classes consisting of the archimidean norms, those primes over 2, and any primes dividing a coefficient of $h$).
For any $s \in S$, as $f = g - h$ is isotropic in $F_s$ there is some $x_s \in F_s$ such that $g,h$ both represent $s$. 
By the Approximation Theorem (\cite{ANT}, II.3.4) we can find $x \in F$ such that $||x -x_s|| < \epsilon_s$ (here we use the multiplicative valuatino) for all $s$ and arbitrarily small $\epsilon_s$.
As the group of squares in each $F_s$ is open (by \Cref{open}) there exist small enough $\epsilon'_s$ such that $x/x_s \in (F_s^{\times})^2$ whenever $||x - x_s|| < \epsilon_s$ for all $s$.
However, note that that for $x_1, x_2 \in F$ that
\[||g(x_1, x_2)|| \leq a_1||x_1||^2 + a_2 ||x_2||^2\]
then from the previous constructions, by constraining small enough we can find $x_1, x_2 \in F$ such that $||g(x_1, x_2) - x_s|| < \epsilon_s$ for all $s$ so that $g(x_1, x_2)/x_s \in (F_s^{\times})^2$.
To show that $f = g -h$ is isotropic it then suffices to show that $h$ represents $\alpha$.
For each $s \in S$ we know $h$ represents $x_s$, since $g(x_1, x_2)/x_s \in  (F_s^{\times})^2$ then $h$ also represents $g(x_1, x_2)$.
For $v \in V-S$, $h$ is isotropic so represents all of $F_v^{\times}$, in particular $g(x_1, x_2)$ as well. (\Cref{prop:isotropic-hyperbolic}).
So we have reduced our problem to showing that the quadratic form $g(x_1,x_2)Z^2 - h$ represents 0 if and only if it does at each $F_v, v \in V$.
This is a quadratic form of rank $n -1$ and so reduces to an inductive hypothesis (the base case is when we proved the Hasse Minkowiski Theorem for $n = 4$), completing the proof.

\end{proof}

\begin{corollary}
	A quadratic form $f$ over $F$ represents some $a \in F^{\times}$ if and only if $f$ represents $a$ over each $F_v, v \in V$.
\end{corollary}

\begin{proof}
	Apply the Hasse-Minkowksi Theorem to the form $f - aZ^2$.
\end{proof}

\section{Selmer's Equation}


Suppose we are looking for some solution to an equation over the integers. 
If one suspects there are none, a natural first direction is to look mod $p^n$ for some prime $p$ (or in $\Z_p$) and if there are no solutions there we can conclude there are no "global" solutions.
If this is not possible for any $p$, one may begin to believe that there should be a global solution.
For the case of quadratic forms, this was answered affirmatively as the Hasse-Minkowksi theorem in the previous section.
However, for equations of higher degree there need not be such a local-global principle, one famous example which we cover in this section is Selmer's equation
\[f(X,Y,Z) = 3X^3 + 4Y^3 + 5Z^3\]
which we will show has nontrivial solutions in each $\Q_p$ (and $\R$), but no nontrivial solution over $\Q$, we will roughly follow \todo{Citee}.

\subsection{Local Solutions}

It is easy to find nontrivial solutions over $\R$, for example we have $f(0, \sqrt[3]{ \frac{5}{3}}, -1) = 0$.
We now show how one can construct solutions over each $\Q_p$.
The main idea is to use the variant of Hensel's lemma \Cref{theo:hensel} introduced previously, which allows us to lift solutions mod from $\Z/p^n\Z$ where our life is easier.
We will need to do some casework depending on the prime $p$. \newline

\noindent $p = 3:$ We work in $\Z/27 \Z$, setting $X = 0, Z = -1$ we wish to find a zero to $g(Y) = 4Y^3 - 5$ where , for which one verifies that $Y = 2$ works.
Now $v_3(g'(2)) = v_3(48) = 1$ while $v_3(g(2)) = v_3(27) = 3$ so it satisifies the requirements of $\Cref{theo:hensel}$ and lifts to a solution $a$ of $4Y^3 - 5$ over $\Q_3$.
Then $f(0,a,-1) = 0$ is a solution of Selmer's equation over $\Q_3$. (Note the factor of $3$ in $g'$ is why we needed to work over $\Z/27 \Z$ instead of a lower power of 3). \newline

\noindent $p = 5:$ We work in $\Z/5\Z$, setting $Y = 1$ we are looking for a zero of $g(X) = 3X^3 + 4$, for which $Y = 3$ works.
Now $v_5(g'(Y)) = v_5(81) = 0$, so our solution lifts from Hensel's lemma to a solution $a$ over $\Q_5$, in which case $f(a, 1, 0) = 0$ is a solution of Selmer's equation over $\Q_5$.\newline

\noindent Now suppose that $p \neq 3,5$ is a prime, such that $3$ is a cube in $\Z/p\Z$.
Then this zero $c$ of $g(X) = X^3 - 3$ in $\Z/p\Z$ always lifts to a cube root of 3 $a$ over $\Q_p$ by Hensel's as $v_p(g'(a)) = 0$ as $p \neq 3$, in which case $f(a, -1, -1)$ is a solution of Selmer's equation over $\Q_p$. \newline

\noindent The only primes we need to consider now are those $p \neq 3,5$ where $3$ is not a cube.
Note that we must have $p \equiv 1 \bmod {3}$, as otherwise $(\Z/p\Z)^{\times}$ is cyclic of order relatively prime to 3, in which case the cube map is surjective so every number (hence 3) would be a cube.
Then if $p \equiv 1 \bmod {3}$, in which case $(\Z/p\Z)^{\times}$ is cyclic of order divisible by 3, the subgroup of cubes has index 3.
By assumption, 3 and $3^2 = 9$ are coset representatives for the remaining two non-cube cosets.
We now finish our work by doing casework on which coset 5 is in for each $p$.
\begin{enumerate}
	\item If $5 \equiv x^3 \pmod{p}$, then as $p \neq 3$ we can lift the solution of $X^3 - 5$ by Hensel's to one a solution $a$ over $\Q_p$, in which case $f(-a, a, -1)$ is a solution of Selmer's equation over $\Q_p$. 
	\item If $5 \equiv 3 x^3 \pmod{p}$, this similarly lifts to a solution $a$ of $\sqrt[3]{5/3}$ over $\Q_p$, in which case $f(a, 0, -1)$ is a solution of Selmers' equation over $\Q_p$. 
	\item If $5 \equiv 9 x^3 \pmod{p}$ then $15 \equiv (3x)^3 \pmod{p}$ is a cube and so lifts to a solution $a$ if $\sqrt[3]{15}$ over $\Q_p$, in which case we may verify that
		\[f(3a, 5, -7) = 81(15) + 4(125) - 5(343) = 0\]
		is a solution of Selmer's equation over $\Q_p$.
\end{enumerate}

\noindent And so we have shown that over $\R$ and all $\Q_p$ that Selmer's equation has a nontrivial solution.

\subsection{The Number Field $\Q(\sqrt[3]{6})$}
To show that Selmer's equation has no global solution over $\Q$, we will examine our equation over the number field $\Q(\alpha)$ where $\alpha = \sqrt[3]{6}$, so we will begin beforehand by establishing some basic facts about this number field. 

\begin{proposition}
	The ring of integers of $\Q(\alpha)$ is $\Z[\alpha]$.
\end{proposition}

\begin{proof}
	We have that 
	\[|\text{disc}(\Z[\alpha])| = |N_{\Q}^{Q(\alpha)}(3\alpha^2)| = 27(6^2) = [\mc{O}_{Q(\alpha)}: \Z[\alpha]]^2 \text{disc}(\mc{O}_{Q(\alpha)})\]
	Thus the only prime factors dividing the index $[\mc{O}_{Q(\alpha)}: \Z[\alpha]]$ are 2 and 3.
	Denoting $\Z[\alpha] = A$ for simplicitly, note that in general we have 
	\[[\mc{O}_{Q(\alpha): A}] = \prod_{\text{$p$ prime}} [\mc{O}_{Q(\alpha)_p}: A_p]\]
	for $\mc{O}_{Q(\alpha)_p}$ and $A_p$ the completions at $p$.
	We have $\mc{O}_{Q(\alpha)}/A_p \cong \prod_{i =1}^n \Z/p_i^{n_i} \Z$ for primes $p_i$ dividng the index and corresponding $n_i$, then as tensor product always commutes with quotients, 
	\[\mc{O}_{Q(\alpha)_{p_j}}/ A_{p_j} = \mc{O}_{Q(\alpha)} \otimes \Z_{p_j}/ A \otimes \Z_{p_j} = (\mc{O}_{Q(\alpha)}/A) \otimes \Z_{p_j} = \Z/p_j^{n_j}\]
captures just the component corresponding to $p_j$.
However, since $X^3 - 6$, the minimal polynomial of $\alpha$ is Eisenstein for $p = 2,3$, we know the image of $\alpha$ in $\Q(\alpha)_2$ and $\Q(\alpha)_3$ is an uniformizer, and hence $\mc{O}_{Q(\alpha)_{p}} = (\Z[\alpha])_p$ for $p = 2,3$ shows the index must in fact be 1, hence $\mc{O}_{\Q(\alpha)} = \Z[\alpha]$.
\end{proof}

We now examine how certain primes split in $\Q(\alpha)$.
As we verified the ring of integers is $\Z[\alpha] \cong \Z[x]/(x^3 - 6)$, the splitting of primes over $p$ depends on the splitting of $(x^3 - 6)$ in the fibers $\mb{F}_p[x]/(x^3 - 6)$.
Since $x^3 - 6 = x^3$ over $\mb{F}_2[x]$ and $\mb{F}_3[x]$ there exists unique ramified primes $\mk{p}_2$ and $\mk{p}_3$ of $\Q(\alpha)$ lying over $(2)$ and $(3)$ with norms 2 and 3, respectively.
Now in $\mb{F}_5[x]$ we have that $x^3 - 6 = x^3 - 1$ factors into irreducibles as $(x-1)(x^2 + x + 1)$, so there are two primes $\mk{p}_5$ and $\mk{p}_{25}$ of $\Q(\alpha)$ corresponding norms lying over 5.
In $\mb{F}_7[x]$ we have that $x^3 - 6 = x^3 + 1$ factors into irreducibles as $(x+1)(x+2)(x+4)$ (the roots being the negatives of the cube roots of unity) so there are three primes $\mk{p}_7, \mk{p}'_7$ and $\mk{p}''_7$ of $\Q(\alpha)$ lying over $7$.
Furthermore we must have $\mk{p}_7 | (\alpha + 1), \mk{p}'_7 | (\alpha + 2)$ and $\mk{p}''_7 | (\alpha + 4)$.
\newline

\indent We can establish a little more. 
First note that if $z$ is an integer, then $N_{Q}^{Q(\alpha)}(\alpha + z) = z^3 + 6$.
Since $|N_{Q}^{Q(\alpha)}(\alpha - 2)| = |-2| = 2$, $|N_{Q}^{Q(\alpha)}(\alpha - 1)| = 5$, and  $|N_{Q}^{Q(\alpha)}(\alpha - 1)| = |-5| = 2$ are also the norms of their principal ideals, we have $\mk{p}_2 = (\alpha - 2)$ and $\mk{p}_5 = (\alpha - 1)$.
Now $|N_{Q}^{Q(\alpha)}(\alpha+1)| = 7$ so similarly $(\alpha + 1) = \mk{p}_7$, and $|N_{Q}^{Q(\alpha)}(\alpha + 2)| = 14, N_{Q}^{Q(\alpha)}(\alpha + 4)| = 70$ so $(\alpha + 2) = \mk{p}'_7 \mk{p}_2$ and $(\alpha + 4) = \mk{p}''_7\mk{p}_2\mk{p}_5$.
We are now in good position to establish the following proposition.

\begin{proposition}
	$\Q(\alpha)$ has class number 1, in particular $\Z[\alpha]$ is a PID.
\end{proposition}

\begin{proof}
	$\Q(\alpha)$ has only one pair of complex embeddings.
	Our Minkowski bound is thus 
	\[ \frac{4}{\pi} \frac{3!}{3^3} \sqrt{|\text{disc}(\Z[\alpha])|} = \frac{4}{\pi}\frac{6}{27} \sqrt{972} \approx 8.8\]
	so any element of the class group has representative of norm $\leq 8$.
	Given the unique prime $\mk{p}_2 = (\alpha -2)$ over $2$ with norm 2, the only ideals of norm 2,4,8 are the corresponding multiples and are principal.
	Since $|N_{Q}^{Q(\alpha)}(\alpha)| = 6$ we have $(\alpha) = \mk{p}_2 \mk{p}_3 = (\alpha -2) \mk{p}_3$ shows the unique prime over 3 (with norm 3) is also principal. 
	We showed that $\mk{p}_5 = (\alpha - 1)$ is principal.
	Finally, all that remains to check are the ideals of norm 7.
	But we had $(\alpha + 1) = \mk{p}_7$, $(\alpha + 2) = \mk{p}'_7 \mk{p}_2 = \mk{p}'_7(\alpha - 2)$ and $(\alpha + 4) = \mk{p}''_7\mk{p}_2\mk{p}_5 = \mk{p}''_7 (\alpha - 2) (\alpha - 1)$ means that $\mk{p}_7, \mk{p}'_7$ and $\mk{p}''_7$ are all principal as well.
	Hence the only class group are those principal ideals as claimed.
\end{proof}

As a final prerequisite we establish some facts regarding the group of units of $\Z[\alpha]$.

\begin{proposition}
	The units in $\Z[\alpha]$ (modulo the cubed units) have as representatives $(1 - 6\alpha + 3\alpha^2)^k$ for $k = 0,1,2$.	
\end{proposition}

\begin{proof}
	Given the one real embedding and one pair of complex embeddings of $\Q(\alpha) \hookrightarrow \C$, we have by Dirichlet's Unit Theorem that the group of units is isomorphic to $\{\pm 1\} \times \Z$.
	$1$ and $-1$ are clearly both cubes, so the group of units modulo cubes is a subgroup of index 3. 
	So any noncube unit has its powers as representatives modulo the cubes.
	Since $(2) = \mk{p}_2^3 = (\alpha - 2)^3$ we have that 
	\[ \frac{(2 - \alpha)^3}{2} = \frac{8 - 12\alpha + 6\alpha^2 - \alpha^3}{2} = 1 - 6\alpha + 3\alpha^2 \]
 is a unit.
 It is not a cube, as modulo $\mk{p}_7 = (\alpha + 1)$ where $\Z[\alpha]/\mk{p}_7 \cong \Z/7\Z$ we have 
\[1 - 6\alpha + 3\alpha^2 = 1 - 6(-1) + 3(1) = 3 \pmod{ \mk{p}_7 }\]
is not a cube, hence $1 - 6\alpha + 3\alpha^2$ isn't one in $\Z[\alpha]$. 
\end{proof}

\subsection{Global Solutions}


We finally begin establishing the non-existence of global solutions for Selmer's equation. 
We start with a series of simplifications.
Multiplying Selmer's equation by 2, we may rearrange it as
\[(2X)^3 + 6Y^3 = 10(-Z)^3\]
and there is a clear bijection between rational solutions of Selmer's equation and those of 
\begin{equation}\label{eq1}
	X^3 + 6Y^3 = 10Z^3
\end{equation}


Focusing on the latter equation, for any nontrivial solution $(x,y,z)$ we may by its homogeneity multiple through by coefficient and assume that \textbf{$x,y,z$ are integers}.
Note that $6,10$ are not cubes, so if some prime dividing two of $(x,,y,z)$ it would have to divide the other.
Dividing through we may assume that \textbf{$x,y,z$ are relatively prime}.
If either one of $x,y,z$ were zero, we would have equality of the two reamining cubes, but as $6,10$ are again not cubes this implies the remaining two are also zero, so \textbf{$x,y$ and $z$ must all be nonzero}.
Looking modulo 2 we have that \textbf{$x$ must be even}.
If $y$ or $z$ were also even, the other must be as 2 divides 6 and 10 only once, so as they are relatively prime we have that \textbf{$x,y$ are both odd}.
Similarly arguing for the primes 3 and 5 we also conclude that \textbf{$x,z$}  are not divisibly by 3 and \textbf{$x,y$ are not divisible by $5$}. \newline

\indent Once again denoting $\alpha = \sqrt[3]{6}$ we may factor \Cref{eq1} as 
\begin{equation}\label{eq2}
	(X+Y\alpha )(X^2 - XY\alpha + Y^2 \alpha^2) = (10)(Z)^3
\end{equation}
viewing the above as (principal) ideals in $\Z[\alpha]$.\\

We now argue that 

\newpage

\bibliographystyle{plain}
\bibliography{lib}
\end{document}
